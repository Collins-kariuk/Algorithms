 \documentclass[11pt]{article}
\usepackage{fullpage}
\usepackage{graphicx}
\usepackage{url}
\usepackage{clrscode}
\usepackage{amsmath}


\title{CS140 - Assignment 0\\\small{Due: Sunday,  Jan 21 at 10pm}}
\author{}
\date{}


\begin{document}
\maketitle

\begin{center}

\footnotesize{http://xkcd.com/1162/}
\end{center}


\begin{itemize}
\item For this assignment you must work with a partner.   You can work with someone from either section.  If you would like help finding a partner, let me know \textbf{ASAP}.

\item This assignment must be typeset using \LaTeX.  You are
  encouraged to take the source file for the assignment and modify it by adding your solutions.  
  
\item An important part of being a computer scientist is the ability
  to express solutions clearly and thoroughly.  Therefore, you are
  expected to explain each step of your solution and to present your
  solutions clearly and precisely.  Part of the score on each problem
  will be for quality of presentation.  Note that correct answers
  without justification are not worth very many points!
\end{itemize}

\vspace*{.1cm}

\begin{enumerate}

\item \textbf{[14 points]} Which is bigger?

For each of the two options below, state which one is larger (or if they're equal) and give a justification for your answer.  If the base of the log is not specified, then the answer should apply to all bases ($\ge 2$).  Your justification should be similar to those we looked at in class.  Do \textbf{not} simply plug these answer into a calculator (though you're welcome to check your logic that way).  For all variables (e.g., $x$ or $n$, assume they are positive).

\begin{enumerate}

\item $\log 10$ vs. $\log 20$

    $\log 20$ because log function is increases with increasing with increase in the argument. That is, the log is an increasing function.

\item $\log_4 n$ vs $\log_5 n$

    $\log_4 n > \log_5 n$ \\
    $\log_4 n$ = $\frac{log n}{log 4}$ \\
    $\log_5 n$ = $\frac{log n}{log 5}$ \\
    this holds because maintaining $n$ as the numerator, the $log 4 < log 5$


\item $\log_5 75$ vs $\log_2 12$

    $\log_5 75 > \log_2 12$ \\
    $\log_5 75 = \log_5 5 + \log_5 5 + \log_5 3  = 2 + \log_5 3 $\\
    $\log_2 12 = \log_2 2 + \log_2 2 + \log_2 3 2 + \log_2 3 $\\
    
    $\log_2 3 > \log_5 3$ by 1(b) above.


\item $\log_7 24.5 $ vs $\log_5 12.5$

    $\log_7 24.5 = \log_7 (\frac{49}{2}) = \log_7 49 - \log_7 2 = 2 - \log_7 2$ \\
    $\log_5 12.5 = \log_5 (\frac{25}{2}) = \log_5 25 - \log_5 2 = 2 - \log_5 2$ \\
    but $\log_5 2 > \log_7 2$ by 1(b) above, hence $\log_7 24.5 > \log_5 12.5$ \\


\item $\log x^4$ vs $\log x + \log x^3$

    $\log x^4$ = $\log x + \log x^3$ \\
    $\log x^4 = 4 \log x$ \\
    $\log x + \log x^3$ = $\log x + 3\log x = 4 \log_x$ as above


\item $((x^2)^2)^2$ vs $x^7$

     $((x^2)^2)^2 > x^7$ \\
      $((x^2)^2)^2 = x^(2\cdot2\cdot2) = x^8 > x^7$


\item $a^b$ vs. $b^a$ for constants $a$ and $b$, where $a > b > 3$.

    $b^a > a^b$ \\
    By way of examples, we were able to figure out that the value of the exponent matters more than the value of the base so $b^a > a^b$.

\end{enumerate}


\item \textbf{[5 points]} Pseudocode

\begin{enumerate}

\item Write pseudocode for a function \texttt{dedup} that takes as input an array/list and returns a new array/list with all of the duplicates removed.  You don't have to necessarily follow the conventions used in class, but use something reasonable and be consistent. 
\begin{codebox}
    \Procname{$\proc{DEDUP}(A)$}
    \li $a \gets []$
    \li $b \gets \{\}$
    \li \For $i \gets 1$ \To $\id{length}[A]$
    \li	\If $A[i]$ not in  $b$
    \li \hspace{1em} add $A[i]$ to $a$ 
    \li \hspace{1em}  add $A[i]$ to $b$ 
        \End
     \End
    \li \Return $a$
       \end{codebox}

\item What is the big-O running time of your function?  Give a brief (one sentence) justification.

    \textbf{Answer:}
    
    O(n) since the function runs a single for-loop 
\end{enumerate}


\item\textbf{[9 points]} Properties of Logs

To get you warmed up, here is an example proof showing that $\log_{b} xy = \log_{b} x + \log_{b} y$.

Let:

$k = \log_{b} xy$\\
$\ell = \log_{b} x$\\
$m = \log_{b} y$

We want to show that $k=\ell+m$.\\

- By the definition of logarithms, we know:

$b^{k} = xy$\\
$b^{\ell} = x$\\
$b^{m} = y$\\

- From these, by properties of exponents

$b^{k} = b^{\ell} b^{m} = b^{\ell + m}$\\

Taking the log of both sides, we obtain $k = \ell + m$, which is what we wanted to show.\\

Now give proofs for each of the following properties of logarithms.
Write your proofs out carefully.  You should assume that $a, b, c, n$
are positive \emph{real numbers} (not necessarily integers).

\begin{enumerate}
\item $\log_{b} a^{n} = n \log_{b} a$

    Let:\\
    $k=log_b a^n$\\
    $l=log_b a$\\
    We want to show that $k=nl$

    - By the definition of logarithms, we know:
    $b^k=a^n$ and $b^l=a$

    - We can apply exponent to the nth power to $b^l=a$ to get:

\begin{align*}
    (b^l)^n &= a^n \\ 
    b^{ln} &= a^n  \quad \text{using the exponential power laws.} \\ 
    b^{ln} &= b^k \quad \text{since} \quad b^k=a^n \\
    ln &= k \quad \text{taking the log of both sides.} \\
    k &= ln \quad \text{which is what we wanted to show.} \\
\end{align*}

\item $\log_{b} a = \frac{\log_{c} a}{\log_{c} b}$
    - Let:\\
    $k=log_b a$\\
    $l=log_c a$\\
    $m=log_c b$
    
    We want to show that $k = \frac{l}{m}$.

    - By the definition of logarithm:
    
    $b^k=a$\\
    $c^l=a$\\
    $c^m=b$\\

    - Therefore $a=b^k=c^l$. Concentrating on the last two:
    \begin{align*}
        b^k&=c^l \\
        (c^m)^k &= c^l \quad \text{since} \quad c^m=b \\
        c^{mk} &= c^l \quad \text{using the exponential power laws.} \\ 
        mk &= l \quad \text{taking the log of both sides.} \\
        k &= \frac{l}{m} \quad \text{which is what we wanted to show.} \\
    \end{align*}


\item $a^{\log_{b} n} = n^{\log_{b} a}$

Take the log of both sides \\
\begin{align*}
    \log a^{\log_{b} n} &= \log n^{\log_{b} a} \\
    \log_{b} n \log a &= \log_{b} a \log n \quad \text{using the property in (a).}\\
    \text{Focusing on the left-hand side:} \\
    \log_{b} n \log a &= \frac{\log n \cdot \log a}{\log b} \quad \text{using the property in (b).}\\
    \text{Focusing on the right-hand side:} \\
    \log_{b} a \log n &= \frac{\log a \cdot \log n}{\log b} \quad \text{using the property in (b).}\\
    \text{Equating both sides, we find that they are equivalent} \\
    \frac{\log a \cdot \log n}{\log b} &= \frac{\log n \cdot \log a}{\log b}
\end{align*}

\end{enumerate}

\vspace*{.4cm}

\item \textbf{[5 points]} Running Times 

Suppose you have algorithms that execute the following number of
operations as a function of the input size $n$.  If you have a
computer that can perform $10^{10}$ operations per second, for each
algorithm what is the largest (integer!) input size $n$ for which you
would be able to get the result within a minute?  Be as precise as
possible.  But, for once, no explanation necessary!
\begin{enumerate}
\item $60n^2$ 

    $n=100000$
\item $n^3$

    $n=8434$

\item $\sqrt{n}$

    $n=3.6 \times 10^{23}$
\item $n\log_2 n$

    $n=1.763\times 10^{10}$
    
\item $2^n$

    $n = 39$

\end{enumerate}

\vspace*{.5cm}

\item \textbf{[20 points]} Writing proofs

  The objective of these two problems are to reinforce clear and precise
writing on mathematical material.

\begin{enumerate}
\item
There are two buses, A and B, about to take students on a field trip.
Bus A contains 50 1st grade students.  Bus B contains 50 2nd grade
students.  Before the buses leave, 8 students run out of Bus A and
onto Bus B.  The teachers then randomly choose 8 of the now 58
students on Bus B and force them to move to Bus A.  The buses then
(finally!) drive off.

Are there more 2nd grade students in Bus A or more 1st grade students
in Bus B?

\textbf{Answer:}

Let's denote the number of second graders selected to move from B to A as $x$. Consequently, the number of first graders moving from B to A would be $8-x$. This means that, in the end, A contains $x$ second graders. To determine the final number of first graders in B, we consider that initially 8 first graders moved from A to B and later $8-x$ of them moved back to A, leaving $8 - (8-x) = x$ first graders in B. Therefore, it follows that there are equal numbers of first and second graders in both A and B.


\item Prove by induction that $\sum_{i=1}^n \frac{1}{i(i+1)} =
  \frac{n}{n+1}$ for all integers $n \geq 1$.  Do you need strong
  induction?  Why or why not?

  \textbf{Answer:}

  We show that  $\sum_{i=1}^n \frac{1}{i(i+1)} =
  \frac{n}{n+1}$ for all integers $n \geq 1$.

  1. \textbf{Base case}: We show that the statement holds for $n=1$.
  $\sum_{i=1}^1=\frac{1}{1+1}=\frac{1}{2}$. If we do the same for the left hand side, $\sum_{i=1}^1 = \frac{1}{1(1+1)}=\frac{1}{2}$. They're both equal, as we would expect, and so the base case holds true.
  
  2. \textbf{Inductive step}.
    The inductive hypothesis: We assume that the statement is true for some integer $k$, $\sum_{i=1}^k=\frac{1}{i(i+1)}=\frac{k}{k+1}$.
    
    Using the inductive hypothesis, we need to prove $\sum_{i=1}^{k+1}=\frac{1}{i(i+1)}=\frac{k+1}{k+2}$

    By the definition of sum:

    $\sum_{i=1}^{k+1}=\frac{1}{i(i+1)}=\frac{1}{(k+1)(k+2)}+\sum_{i=1}^k\frac{1}{i(i+1)}$

    By the inductive hypothesis:

    \begin{align*}
        \sum_{i=1}^{k+1} &= \frac{1}{(k+1)(k+2)}+\frac{k}{k+1} \\
        &= \frac{k(k+2)+1}{(k+1)(k+2)} \\
        &= \frac{k^2+2k+1}{(k+1)(k+2)} \\
        &= \frac{(k+1)^2}{(k+1)(k+2)} \\
        &= \frac{k+1}{k+2} \quad \text{This is what we wanted to show}
    \end{align*}

    Finally, we can conclude that $\sum_{i=1}^n \frac{1}{i(i+1)} =
  \frac{n}{n+1}$ for all integers $n \geq 1$.

  Strong induction is not necessary in this case, as we have successfully demonstrated that weak induction is entirely adequate to establish the validity of the given statement.
    
    
    
\end{enumerate}


\end{enumerate}

\vspace*{.4cm}

\end{document}

